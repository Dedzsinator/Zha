\chapter*{Abstract}

Ez a dolgozat a mélységi tanulás módszereinek alkalmazását vizsgálja szimbolikus zenei generálásban, különös tekintettel a Markov-láncok, Variacionális Autoenk-oderek (VAE) és Transformer modellek integrációjára. A kutatás középpontjában a Zha rendszer áll, amely egy hibrid architektúrát alkalmaz a zenei kreatív folyamat modellezésére.

A dolgozat részletesen elemzi az egyes modellek elméleti alapjait és gyakorlati implementációját. A Markov-láncok zenetudományi integrációját, a VAE-k reparameterizációs technikáit és a Transformer modellek memória-alapú generálási stratégiáit mutatjuk be. Különös figyelmet fordítunk a strukturált zenei generálásra, a hosszú távú koherencia biztosítására és a számítási hatékonyságra.

A Zha rendszer innovatív megközelítése többrétegű AI architektúrán alapul, amely FastAPI mikroszolgáltatásokat, fejlett training infrastruktúrát és interaktív web felületet kombinál. Az empirikus értékelés során azt találtuk, hogy a hibrid megközelítés szignifikánsan javítja a generált zene minőségét és diverzitását a hagyományos egymodelles megközelítésekhez képest.

A kutatás hozzájárul a számítógépes zenei kreativitás területéhez azáltal, hogy bemutatja a különböző mélységi tanulási paradigmák hatékony integrálását és a zenei struktúra komplex modellezési kihívásainak megoldását.

\textbf{Kulcsszavak:} mélységi tanulás, zenei generálás, Transformer, VAE, Markov-láncok, hibrid modellek